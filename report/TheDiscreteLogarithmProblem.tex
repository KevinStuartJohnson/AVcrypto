% TODO
% Describe DSA
% Mension attacks 

Given an arbitrary finite cyclic group $G$ with group operation $\cdot$ and generator $g \in G$, discrete exponentiation by $a$ in $G$ is defined by 

$$
g^a = \overbrace{g \cdot g \cdots g}^{a \text{ times}}
$$

If $y = g^a$ is known, computing $a$ is called finding the discrete logarithm of $y$. With the method of fast exponentiation, $y$ can be computed quickly, only $O(\text{log } a)$ group operations. On the other hand, computing $a$ can be much harder. In fact, in [\ref{VictorShoup}] it was shown that in an arbitrary group for which only the group operation and discrete exponentiation can be applied to group elements, computing discrete logarithms will take at least $O(\sqrt{\mid G \mid})$ operations. In most cases though, more structure is known about the group in use.  

\subsection{The Diffie-Hellman Key Exchange}

This one-way property of discrete exponention has proven to be very useful for cryptographic purposes. The most notable of these is in the Diffie-Hellman Key Exchange Protocal in which two parties $A$ and $B$ wish to share a secret key $k$.

\begin{algorithm} 
	\caption{Diffie-Hellman Key Exchange Protocal}
	\begin{algorithmic}[1]
		\State $A$ and $B$ share a publicly known group $G$ and generator $g$.
		\State $A$ chooses a random private exponent $a$ and computes $g^a$.
		\State $B$ chooses a random private exponent $a$ and computes $g^b$.
		\State $A$ sends $g^a$ to $B$ and $B$ sends $g^b$ to $A$. 
		\State $A$ raises $g^b$ to their own private exponent $a$ to obtain $k = (g^b)^a = g^{ab}$.
		\State $B$ raises $g^a$ to their own private exponent $b$ to obtain $k = (g^a)^b = g^{ba}$.
	\end{algorithmic} 
\end{algorithm}  


The two parties may now use $k$ to communicate with a cryptographically secure communication protocol. The described protocol relies on the hardness of computing $g^{ab}$ given $g^a$ and $g^b$, which is conjectured in [\ref{WhitfieldDiffieMartinHellman}] to be equivalent to computing discrete logarithms.


\subsection{The Digital Signature Algorithm}

In many web based security protocols, it's important to have a scheme for demonstrating the authenticity of a digital message or document. In 1976, Whitfield Diffie and Martin Hellman described a solution to this problem with concept of \textit{digital signatures}. Further information on digital signatures can be found in [\ref{DougStinson}]. We describe the Digital Signature Algorithm (DSA) which is the US Federal Information Processing Standard for digital signatures. \\ 

Let $G$ be a finite cyclic group of order $N$ with generator $g$. Suppose party $A$ wants to send a signed message to a party $B$. {\color{blue}  Finish describing algorithm}


\begin{algorithm} 
	\caption{DSA }
	\begin{algorithmic}[1]
		\State 
	\end{algorithmic} 
\end{algorithm} 


\subsection{A Brief History of the Groups $\mathbb{F}_p^*$ and $\mathbb{F}_{2^n}^*$}

Given a finite field $\mathbb{F}$, the multiplicative units $\mathbb{F}^* = \mathbb{F} \backslash \lbrace 0 \rbrace $ form a finte cyclic group and thus may be used for discrete logarithm based cryptography. It is a standard result from algebra that every finite field has order $p^n$, where $p$ is a prime and $n \in \Z^+$. We divide the discussion of the cryptographic properties of the group $\mathbb{F}_{p^n}^*$ into two cases, when $n = 1$ and when $n>1$. \\

In the later case, when working with finite fields of order $p^n$, the arithmetic is only really tractable when $p=2$. In the 1980's researches at the University of Waterloo made attempts to construct discrete logarithm cryptosystems based on $\mathbb{F}_{2^{127}}^*$ which initially paralelled RSA in terms of bits of security. But in 1986, Don Coppersmith devised an astonishing algorithm in [\ref{DanCoppersmith}] which could compute discrete logarithms in the group $\mathbb{F}_{2^{127}}^*$ in about $5$ minutes. Further attempts were made too increase the size to $n=593$ but similar adaptations of Coppersmith's algorithm made researchers abandone public key cryptosystems based on the discrete logarithm problem in $\mathbb{F}_{2^n}^*$. \\

When $n = 1$, we have a group which is essentially just the non-zero integers mod a prime. As one might expect, when $p$ is small, computing discrete logs in $\mathbb{F}_p^*$ can be done quickly with just trial exponentiation. When $p$ is large though, say $p = 2^{1000}$, this method becomes completely intractable, even on todays fastest computers. That being said, there is an attack described in [\ref{NFStoDLP}] which adapts the Number Field Sieve to solve discrete logs in $\mathbb{F}_p^*$. This attack has running time very similar to factoring $$O(p) = e^{1.923(\log p)^{1/3}(\log\log p)^{2/3}}$$ This basically means that the bit length required for $p$ in $\mathbb{F}_{p^n}^*$ based cryptosystems is the same as the bit length required for the modulus in RSA. In todays standards that mean $p = 2^{2048}$. Although this cryptographically viable, in practice using such large values of $p$ has its limitations. Such as bandwidth in network communications or memory in a hand-held devices. \\

These two case made researches search for alternative groups with cryptographically strong properties. 








