Given an arbitrary finite cyclic group $\mf{G}$ with group operation $\cdot$ and generator $g \in \mf{G}$, discrete exponentiation by $a$ in $\mf{G}$ is defined by 

$$
g^a = \overbrace{g \cdot g \cdots g}^{a \text{ times}}
$$

If $y = g^a$ is known, computing $a$ is called finding the discrete logarithm of $y$. With the method of fast exponentiation, $y$ can be computed quickly, only $O(\text{log } a)$ group operations. On the other hand, computing $a$ can be much harder. In fact, in [\ref{VictorShoup}] it was shown that in an arbitrary group for which only the group operation and discrete exponentiation can be applied to group elements, computing discrete logarithms will take at least $O(\sqrt{\mid \mf{G} \mid})$ operations. In most cases though, more structure is know about the group in use. 

\subsection{The Diffie-Hellman Key Exchange}

This one-way property of discrete exponention has proven to be very useful for cryptographic purposes. The most notable of these is in the Diffie-Hellman Key Exchange Protocal in which two parties $A$ and $B$ wish to share a secret key $k$. 

\begin{enumerate}[1.]
	\item $A$ and $B$ share a publicly known group $\mf{G}$ and generator $\mf{g}$. 
	\item $A$ chooses a random private exponent $a$ and computes $g^a$.
	\item $B$ chooses a random private exponent $a$ and computes $g^b$.
	\item $A$ sends $g^a$ to $B$ and $B$ sends $g^b$ to $A$. 
	\item $A$ raises $g^b$ to their own private exponent $a$ to obtain $k = (g^b)^a = g^{ab}$.
	\item $B$ raises $g^a$ to their own private exponent $b$ to obtain $k = (g^a)^b = g^{ba}$.
\end{enumerate}

The two parties may now use $k$ to communicate with a cryptographically secure communication protocol. The described protocol relies on the hardness of computing $g^{ab}$ given $g^a$ and $g^b$, which was conjectured in [\ref{WhitfieldDiffieMartinHellman}] to be equivalent to computing discrete logarithms. \\ 

Another nice property about this protocol is it's applicable to any finite cyclic group. The simplist example of this is in the multiplicative units of the integers modulo a prime $\mathbb{Z}_p^*$. Let $g$ be a primite root mod $p$, then every element of $y \in \mathbb{Z}_p^*$ can be written in the form $y = g^a$ for some $a < p-1$. Thus $\mathbb{Z}_p^*$ is a finite cyclic group and one can compute discrete logarithms in $\mathbb{Z}_p^*$. 


% So great that's all we need? 



