\documentclass[10pt,a4paper]{report}
\usepackage{
  amsmath,amsfonts,euscript,enumerate,mathrsfs,
  hyperref,amsthm,amssymb,upref,graphics,color
  }
\usepackage[all]{xy}
\hypersetup{
  colorlinks,breaklinks,urlcolor=blue,linkcolor=black
  }

%%%%%%%%%%%%%%%%%%%%%%%%%%%%%%%%%%%%%%%%%%%%%%%%%%%%%%%%%
% Theorems 
%%%%%%%%%%%%%%%%%%%%%%%%%%%%%%%%%%%%%%%%%%%%%%%%%%%%%%%%%
% \swapnumbers %% numbering for theorems will be on the left
\theoremstyle{plain} 
  \newtheorem{theorem}[subsection]{Theorem}
  \newtheorem*{thmnonumber}{Theorem}
  \newtheorem{proposition}[subsection]{Proposition}
  \newtheorem{lemma}[subsection]{Lemma}
  \newtheorem*{lemmanonumber}{Lemma}
  \newtheorem{corollary}[subsection]{Corollary}
  \newtheorem*{cornonumber}{Corollary}
  \newtheorem{nothing}[subsection]{}
  \newtheorem{sheafificationThm}[subsection]{Sheafification Theorem}
  \newtheorem{subtheorem}{Theorem}[subsection]
  \newtheorem{subproposition}[subtheorem]{Proposition}
  \newtheorem{sublemma}[subtheorem]{Lemma}
  \newtheorem{subcorollary}[subtheorem]{Corollary}
  \newtheorem{subnothing}[subtheorem]{}

\theoremstyle{definition}
  \newtheorem{definition}[subsection]{Definition}
  \newtheorem{definitions}[subsection]{Definitions}
  \newtheorem*{definonumber}{Definition}
  \newtheorem{nothing*}[subsection]{}
  \newtheorem{example}[subsection]{Example}
  \newtheorem{examples}[subsection]{Examples}
  \newtheorem*{solution}{Solution}
  \newtheorem*{exnonumber}{Example}
  \newtheorem*{quesnonumber}{Question}
  \newtheorem{question}{Question}
  \newtheorem{problem}[subsection]{Problem} 
  \newtheorem{exercise}[subsection]{Exercise} 
  \newtheorem{notation}[subsection]{Notation}
  \newtheorem{notations}[subsection]{Notations}
  \newtheorem{step}{Step}
  \newtheorem*{claim}{Claim}
  \newtheorem{assumptions}[subsection]{Assumptions}
  \newtheorem{subdefinition}[subtheorem]{Definition}
  \newtheorem{subnotation}[subtheorem]{Notation}
  \newtheorem{subnotations}[subtheorem]{Notations}
  \newtheorem{subexample}[subtheorem]{Example}
  \newtheorem{subnothing*}[subtheorem]{}

\theoremstyle{remark}
  \newtheorem*{remark}{Remark}
  \newtheorem*{remarks}{Remarks}
  \newtheorem*{warning}{Warning}
  \newtheorem*{smallexample}{Example}


%%%%%%%%%%%%%%%%%%%%%%%%%%%%%%%%%%%%%%%%%%%%%%%%%%%%%%%%%
%% Macros (Most acronyms come from french)
%%%%%%%%%%%%%%%%%%%%%%%%%%%%%%%%%%%%%%%%%%%%%%%%%%%%%%%%%

\DeclareMathOperator*{\Oplus}{\oplus} 

\newcommand{\coker}{    \operatorname{{\rm coker}}}
\newcommand{\Aut}{    \operatorname{{\rm Aut}}}
\newcommand{\Spec}{   \operatorname{{\rm Spec}}}
\newcommand{\Proj}{   \operatorname{{\rm Proj}}}
\newcommand{\rank}{   \operatorname{{\rm rank}}}
\newcommand{\Reg}{    \operatorname{{\rm Reg}}}
\newcommand{\haut}{   \operatorname{{\rm ht}}}
\newcommand{\supp}{   \operatorname{{\rm supp}}}
\newcommand{\image}{  \operatorname{{\rm im}}}
\newcommand{\ord}{    \operatorname{{\rm ord}}}
\newcommand{\bideg}{  \operatorname{{\rm bideg}}}
\newcommand{\trdeg}{  \operatorname{{\rm trdeg}}}
\newcommand{\Frac}{   \operatorname{{\rm Frac}}}
\newcommand{\Char}{   \operatorname{{\rm char}}}
\newcommand{\Span}{   \operatorname{{\rm Span}}}
\newcommand{\Sing}{   \operatorname{{\rm Sing}}}
\newcommand{\Pic}{    \operatorname{{\rm Pic}}}
\newcommand{\Cl}{   \operatorname{{\rm Cl}}}
\newcommand{\dom}{    \operatorname{{\rm dom}}}
\newcommand{\codom}{  \operatorname{{\rm codom}}}
\renewcommand{\div}{  \operatorname{{\rm div}}}
\newcommand{\id}{   \operatorname{{\rm id}}}
\newcommand{\ob}{   \operatorname{{\rm ob}}}
\newcommand{\Hom}{    \operatorname{{\rm Hom}}}
\newcommand{\Set}{    \operatorname{{\rm\bf Set}}}
\newcommand{\Top}{    \operatorname{{\rm\bf Top}}}
\newcommand{\Grp}{    \operatorname{{\rm\bf Grp}}}
\newcommand{\CRng}{   \operatorname{{\rm\bf CRng}}}

\newcommand{\Mod}[1]{\mbox{\rm ${#1}$-\bf Mod}}
\newcommand{\setspec}[2]{\big\{\,#1\, \mid \,#2\, \big\}}
\newcommand{\powerset}{\raisebox{\depth}{\Large $\wp$}}

\newcommand{\notdiv}{\not\hspace{\mylength}\mid}
\newcommand{\epi}{\twoheadrightarrow}
\newcommand{\overepi}[1]{\overset{ #1 }{\epi}}
\newcommand{\monic}{\rightarrowtail}
\newcommand{\overmonic}[1]{\overset{ #1 }{\monic}}
\newcommand{\Integ}{\ensuremath{\mathbb{Z}}}
\newcommand{\Nat}{\ensuremath{\mathbb{N}}}
\newcommand{\Rat}{\ensuremath{\mathbb{Q}}}
\newcommand{\Comp}{\ensuremath{\mathbb{C}}}
\newcommand{\Reals}{\ensuremath{\mathbb{R}}}
\newcommand{\aff}{\ensuremath{\mathbb{A}}}
\newcommand{\proj}{\ensuremath{\mathbb{P}}}
\newcommand{\bk}{{\ensuremath{\rm \bf k}}}
\newcommand{\ck}{{\bar{\bk}}}
\newcommand{\kk}[1]{\bk^{[#1]}}

\newcommand{\Aeul}{\EuScript{A}}
\newcommand{\Beul}{\EuScript{B}}
\newcommand{\Ceul}{\EuScript{C}}
\newcommand{\Deul}{\EuScript{D}}
\newcommand{\Eeul}{\EuScript{E}}
\newcommand{\Feul}{\EuScript{F}}
\newcommand{\Geul}{\EuScript{G}}
\newcommand{\Heul}{\EuScript{H}}
\newcommand{\Keul}{\EuScript{K}}
\newcommand{\Oeul}{\EuScript{O}}
\newcommand{\Peul}{\EuScript{P}}
\newcommand{\Seul}{\EuScript{S}}

\newcommand{\Acal}{\mathcal{A}}
\newcommand{\Bcal}{\mathcal{B}}
\newcommand{\Ccal}{\mathcal{C}}
\newcommand{\Dcal}{\mathcal{D}}
\newcommand{\Ecal}{\mathcal{E}}
\newcommand{\Fcal}{\mathcal{F}}
\newcommand{\Gcal}{\mathcal{G}}
\newcommand{\OV}{\mathcal{O}}

\newcommand{\pgoth}{\mathfrak{p}}
\newcommand{\Pgoth}{\mathfrak{P}}
\newcommand{\qgoth}{\mathfrak{q}}
\newcommand{\Qgoth}{\mathfrak{Q}}
\newcommand{\m}{\mathfrak{m}}
\newcommand{\M}{\mathfrak{M}}

%%% DEAL With these later
\newcommand{\Q}{\mathbb{Q}}
\newcommand{\R}{\mathbb{R}}
\newcommand{\Z}{\mathbb{Z}}
\newcommand{\C}{\mathbb{C}}
\newcommand{\K}{\mathbb{K}}
\newcommand{\N}{\mathbb{N}}
\newcommand{\FP}{F_P}
\newcommand{\PF}{\mathbb{P}_F}
\newcommand{\LL}{\mathscr{L}} %% RRP 
\newcommand{\Leul}{\mathscr{L}} %% Dimension  of the RRP
\newcommand{\ldim}{\EuScript{l}}
\newcommand{\A}{\mathcal{A}_F}
\newcommand{\notzero}{\backslash \lbrace 0 \rbrace }
\newcommand{\IDEALa}{\mathfrak{a}}

\newcommand{\IDEALp}{\mathfrak{p}}
\newcommand{\fX}{f(X) = a_0 + a_1X + ... + a_nX^n}
\newcommand{\fx}{f(x) = a_0 + a_1x + ... + a_nx^n}
\newcommand{\ket}{\big \rangle}
\newcommand{\bra}{\big \langle}

\newcommand{\Abf}{\mathbf{A}}
\newcommand{\Bbf}{\mathbf{B}}
\newcommand{\Cbf}{\mathbf{C}}
\newcommand{\Dbf}{\mathbf{D}}
\newcommand{\Ebf}{\mathbf{E}}

\newcommand{\PP}{\mathbb{P}}
\newcommand{\PPP}{\mathbb{P}}
\newcommand{\SSS}{\mathbb{S}}
\newcommand{\VV}{\mathbb{V}}
\newcommand{\VVV}{\mathbb{V}}

\newcommand{\dirlim}{\varinjlim}
\newcommand{\ssi}{\Leftrightarrow}
\newcommand{\isom}{\cong}
\renewcommand{\epsilon}{\varepsilon}
\renewcommand{\phi}{\varphi}
\renewcommand{\emptyset}{\varnothing}
\newcommand{\rien}[1]{}
\renewcommand{\baselinestretch}{1.07}
\newcommand{\HRule}{\rule{\linewidth}{0.5mm}} 

%%%%%%%%%%%%%%%%%%%%%%%%%%%%%%%%%%%%%%%%%%%%%%%%%%%%%%%%%
% Unique to this document
%%%%%%%%%%%%%%%%%%%%%%%%%%%%%%%%%%%%%%%%%%%%%%%%%%%%%%%%%

\newcommand{\mf}{\mathfrak}

%%%%%%%%%%%%%%%%%%%%%%%%%%%%%%%%%%%%%%%%%%%%%%%%%%%%%%%%%
% Pages sizing
%%%%%%%%%%%%%%%%%%%%%%%%%%%%%%%%%%%%%%%%%%%%%%%%%%%%%%%%%

% \setlength{\textwidth}{15.5cm}
% % \setlength{\mylength}{1.45\mylength}
% % \settowidth{\mylength}{$\,$}
% \addtolength{\oddsidemargin}{-1cm}
% \addtolength{\evensidemargin}{-1cm}
% \addtolength{\textheight}{14mm}

% \raggedbottom
% \CompileMatrices
% \newlength{\mylength}

\setlength\parindent{0pt}

%% Assignment info 
\begin{document}
{\huge Abelian Variety Cryptosystems} \\
{\large Kevin Johnson 6017605} \\
{\large \today} \\
\HRule  % Horizontal line

% \tableofcontents

\section{The Discrete Logarithm Problem}
  % TODO
% Describe DSA
% Mension attacks 

Given an arbitrary finite cyclic group $\mf{G}$ with group operation $\cdot$ and generator $g \in \mf{G}$, discrete exponentiation by $a$ in $\mf{G}$ is defined by 

$$
g^a = \overbrace{g \cdot g \cdots g}^{a \text{ times}}
$$

If $y = g^a$ is known, computing $a$ is called finding the discrete logarithm of $y$. With the method of fast exponentiation, $y$ can be computed quickly, only $O(\text{log } a)$ group operations. On the other hand, computing $a$ can be much harder. In fact, in [\ref{VictorShoup}] it was shown that in an arbitrary group for which only the group operation and discrete exponentiation can be applied to group elements, computing discrete logarithms will take at least $O(\sqrt{\mid \mf{G} \mid})$ operations. In most cases though, more structure is kno about the group in use.

\subsection{The Diffie-Hellman Key Exchange}

This one-way property of discrete exponention has proven to be very useful for cryptographic purposes. The most notable of these is in the Diffie-Hellman Key Exchange Protocal in which two parties $A$ and $B$ wish to share a secret key $k$. 

\begin{enumerate}[1.]
	\item $A$ and $B$ share a publicly known group $\mf{G}$ and generator $\mf{g}$. 
	\item $A$ chooses a random private exponent $a$ and computes $g^a$.
	\item $B$ chooses a random private exponent $a$ and computes $g^b$.
	\item $A$ sends $g^a$ to $B$ and $B$ sends $g^b$ to $A$. 
	\item $A$ raises $g^b$ to their own private exponent $a$ to obtain $k = (g^b)^a = g^{ab}$.
	\item $B$ raises $g^a$ to their own private exponent $b$ to obtain $k = (g^a)^b = g^{ba}$.
\end{enumerate}

The two parties may now use $k$ to communicate with a cryptographically secure communication protocol. The described protocol relies on the hardness of computing $g^{ab}$ given $g^a$ and $g^b$, which is conjectured in [\ref{WhitfieldDiffieMartinHellman}] to be equivalent to computing discrete logarithms. \\ 

Another nice property about this protocol is it's applicable to any finite cyclic group. The simplist example of this is in the multiplicative units of the integers modulo a prime, denoted $\mathbb{Z}_p^*$. Let $g$ be a primite root mod $p$, then every element of $y \in \mathbb{Z}_p^*$ can be written in the form $y = g^a$ for some $a < p-1$. Thus $\mathbb{Z}_p^*$ is a finite cyclic group and one can compute discrete logarithms in $\mathbb{Z}_p^*$.

% \subsection{The Digital Signature Algorithm}





\section{Algebraic Varieties, Dimension and Genus}
  %TODO
%Proper definition of algebraic variety
%Definition of Dimension 
%Definition of Genus
%Start with algebraically closed field


In this report we focus on groups which arise from the solution set of polynomial equations over finite fields. The most famous of these groups is the set of points on an elliptic curve characterized by the equations $y^2 = x^3 + ax + b$ where $a,b \in \mathbb{F}_p$ satisfy $4a^3 + 27b^2 \not\equiv 0 \text{ mod } p$. In recent years, groups arising from these equations have found significant success in public key cryptography even though it is still an open questions whether these kind of groups are actually cryptographically secure! One of the benefits of using elliptic curves as groups is that the fastest known attack on them has $O(\sqrt{N})$ complexity, where $N$ is the order of the group. This means significantly smaller keys can be used in comparison to the key length of RSA. A natural question is 

$$
\textbf{What about other polynomial equations?}
$$ 

It turns out that there are infinitely many groups which arise from the solution sets of polynomial equations. Which of these groups are cryptographically viable is a vast active area of research. First we develope some notations required to describes these groups. \\ 

Let $K$ be a field. Let $A = K[x_1,...,x_n]$ represent the polynomial ring in $n$ variables over $K$. Given a subset $T \subseteq A$, we may define 

$$
Z(T) = \lbrace P \in K^n \mid f(P) = 0 \text{ for all } f \in T \rbrace 
$$ 

to be the set of all common zeros of polynomials in $T$. We call $Y \subseteq K^n $ an \textit{algebraic subset} if $Y = Z(T)$ for some subset $T \subseteq A$. A not so obvious but usual fact to keep in mind is that if $T \subseteq A$ and $J = \langle T \rangle $ is the ideal generated by elements of $T$, then $Z(T) = Z(J)$. We say an algebraic set $Y$ is reducible if it can be written as the union of two smaller algebraic sets. For example, let $Y = Z(x^2 - yz, xz-x)$ as a subset of $\mathbb{C}^3$. That is, $Y$ is the set of complex valued points in $\mathbb{C}^3$ which satisfy each of the equations $x^2 - yz$ and $xz-x$. Notice that

\begin{align*}
	Y &= Z(x^2 - yz, xz-x) \\
	&= Z(x^2 - yz, x(z-1)) \\
	&= Z(x^2 - yz,x) \cup Z(x^2 - yz,z-1) \\
	&= Z(yz,x) \cup Z(x^2 - yz,z -1) \\
	&= Z(y,x) \cup Z(z,x) \cup Z(x^2 - yz, z - 1)
\end{align*} 

So $Y$ is reducible in $\mathbb{C}^3$. If an algebraic set is not reducible, it is called irreducible. 


% Similary, given an subset $Y \subseteq K^n$, may define the set of A 

% $$ 
% I(Y) = \lbrace f \in A \mid f(P) = 0 \text { for all } P \in Y \rbrace 
% $$ 

% In fact, $I(Y)$ is an ideal in $A$ often called \textit{the ideal of Y}. So we may consider the quotient ring 

% $$
% A(Y) = A / I(Y) 
% $$ 

% We refer to this ring as the \textit{the coordinate ring of Y}. This ring encodes a lot of information about an alebraic set. 





\subsection{Dimension}
\subsection{Genus}



\section{Elliptic Curves}
\section{Hyperelliptic Curves}
  Unlike elliptic curves, when the genus $\mf{g}$ of a curve $\mf{C} $ is greater than $1$, the set of points on $\mf{C}$ will not always form a group. 

\begin{example}

\end{example}

\subsection{The Jacobian of a Hyperelliptic Curve}

Luckily, there is another way to form an abelian group with hyperelliptic curves. Indeed, let $\mf{D}$ be the set of all formal finite sums 

$$ \sum_i m_i P_i $$ 

where $m_i \in \mathbb{Z}$ and $P_i$ are points on the curve $\mf{C}$. We call elements of $\mf{D}$ divisors of $\mf{C}$. Given a rational function $f$ in $\mathbb{Z}_p[\mf{C}]$, we can define the corresponding divisor to $f$ as

$$(f) = \sum_i m_i P_i $$ where $P_i$ are the zeros and poles of $f$ with multiplicities $m_i$. 

\begin{example}
\end{example} 

Divisors of this form are called principal divisors and we let $\mf{P}$ denote the subset of all of them in $\mf{D}$. If we define the operation on $\mf{D}$ by 

$$ \sum_i m_i P_i  + \sum_i m^\prime_i P_i  = \sum_i (m_i+m^\prime) P_i $$ 

then $\mf{D}$ becomes and abelian group. Unfortunetly, this group is far too large and unstructured for cryptographic purposes. So we consider the subgroup $\mf{D}^0$ of all divisors of $\mf{D}$ whos coefficients sum to $0$. That is, divisors $ \sum_i m_i P_i $ such that $\sum_i m_i = 0$. \\ 

This subgroup is still infinite, but that can be remedied by defining two divisors $D_1, D_2$ of $\mf{D}^0$ to be equal if $D_1 - D_2$ is equal to the divisor of a rational function on $\mf{C}$. That is, $D_1 - D_2 = (f) $ for $f \in \mathbb{Z}_p[\mf{C}]$. This new quotient group, denoted $$\mf{J} = \mf{D}^0 / \mf{P}$$ is called the jacobian of the curve $\mf{C}$ and is a finite cyclic group. This will be the group used to build hyperelliptic cryptosystems.

\subsection{Representation of Divisors}

Athough the Jacobian $\mf{J}$ of an hyperelliptic curve $\mf{C}$ is a finite abelian group, elements of $\mf{J}$ are very hard to represent. 

\begin{example}
\end{example} 

% Definition of semi-reduced divisor and how to find it and talk about the weight of a divisor y1report.pdf 

To make the group operation in $\mf{J}$ tractable, we ustilize the mumford representation of a divisor which is described as follows. Let $D$ be a semi-reduced with points $P_i = (x_i,y_i)$. We associate to $D$ polynomials $a,b \in \mathbb{Z}_p[x]$ such that $$a(x) = \prod^r_i (x - x_i) $$ $$ b(x_i) = y_i \text{ } 1 \leq i \leq r $$ where $\deg b < \deg a$ and $(x - x_i)^{k_i} \mid b - y_i$, if $k_i$ is the multiplicity of $P_i$. Denote this representation $D \stackrel{\text{def}}{=} \text{div} (a,b)$.

\subsection{The Group Law}

The group operation can be divided into two parts - \textit{Composition} and \textit{reduction} as described in [\ref{TanjaLange}]. \\ 	


\large{\textbf{Composition}}

Given two divisors represented as $D_1= \text{div}(a_1,b_1), D_2 = \text{div} (a_2,b_2) $ 

\begin{enumerate}[1.]
	\item compute $d_0 = \text{gcd}(a_1,a_2)$ and find the unique $c_1,e_1 \in \mathbb{Z}_p[x]$ such that $d_0 = c_1a_1 + e_1a_2$ 
	\item compute $d = \text{gcd}(d_1,b_1 + b_2)$ and find the unique $c_2, e_2 \in \mathbb{Z}_p[x]$ such that $d = c_2d_1 + e_2(b_1 + b_2)$ 
	\item compute $a_3 = \frac{a_1,a_1}{d^2}$
	\item compute $b_3 = \frac{c_2c_2a_1 + c_2e_1a_2 + e_2(b_1b_2 + f)}{d} \text{ mod } \frac{a_1,a_1}{d^2}$
	\item \label{repeat} compute $a_3^\prime = \frac{f - b_3^2}{a_3}$ and $b_3^\prime = - b_3$ mod $a_3^\prime$ 
	\item while $\deg (a_3^\prime ) > g$, reassign $a_3 = a_3^\prime, b_3 = b_3^\prime$ and repeat step \ref{repeat}
	\item divide $a_3^\prime $ by its leading coefficient so that $a_3^\prime $ becomes monic
	\item the output $div(a_3^\prime,b_3^\prime) = D_1 + D_2$ 
\end{enumerate}


Why Does this work? 

\begin{example}
\end{example}





\section{Abelian Varieties}
\section{Applications}
\section{References}
  %TODO
%Change format to IEEE 

% \item \label{} Author, \textit{Tittle}, Journal, Date, page 

\begin{enumerate}[1]
	\item \label{Hartshorne} Robin Hartshorne, \textit{Algebraic Geometry, Graduate Texts in Mathematics, vol. 52}, Springer-Verlag, New York, 1977, ISBN 0-387-90244-9.   
	
	\item \label{VictorShoup} Victor Shoup, \textit{Lower bounds for discrete logarithms and related problems}, Theory and Application of Cryptographic Techniques, 1997, pp. 256 - 266. 
	
	\item \label{WhitfieldDiffieMartinHellman} Whitfield Diffie and Martin E. Hellman, \textit{New directions in cryptography}, IEEE Transactions on Information Theory \textbf{IT-22} (1976), no. 644-654.  

	\item \label{TanjaLange} Tanja Lange, \textit{Formulae for Arithmetic on Genus 2 Hyperelliptic Curves}, ... not complete.

	\item \label{PohligHellman} Stephan Pohlig and Martin E. Hellman, \textit{An Improved Algorithm for Computing Logarithms over $GF(p)$ and its Cryptographic Significance}, IEEE Transactions on Information Theory, vol. m24, NO. 1, January 1978, pg 106.  

	\item \label{NFStoDLP} An Commeine and Igor Semaev, \textit{An Algorithm to Solve the Discrete Logarithm Problem with the Number Field Sieve}, Public Key Cryptography - PKC 2006, 9th International Conference on Theory and Practice of Public-Key Cryptography, vol, 3958, pg 174-190, 2006. 

	\item \label{DanCoppersmith} Dan Coppersmith, 
\end{enumerate}


\end{document}




