\documentclass[10pt,a4paper]{report}
\usepackage{
  amsmath,amsfonts,euscript,enumerate,mathrsfs,
  hyperref,amsthm,amssymb,upref,graphics,color
  }
\usepackage[all]{xy}
\hypersetup{
  colorlinks,breaklinks,urlcolor=blue,linkcolor=black
  }

%%%%%%%%%%%%%%%%%%%%%%%%%%%%%%%%%%%%%%%%%%%%%%%%%%%%%%%%%
% Theorems 
%%%%%%%%%%%%%%%%%%%%%%%%%%%%%%%%%%%%%%%%%%%%%%%%%%%%%%%%%
% \swapnumbers %% numbering for theorems will be on the left
\theoremstyle{plain} 
  \newtheorem{theorem}[subsection]{Theorem}
  \newtheorem*{thmnonumber}{Theorem}
  \newtheorem{proposition}[subsection]{Proposition}
  \newtheorem{lemma}[subsection]{Lemma}
  \newtheorem*{lemmanonumber}{Lemma}
  \newtheorem{corollary}[subsection]{Corollary}
  \newtheorem*{cornonumber}{Corollary}
  \newtheorem{nothing}[subsection]{}
  \newtheorem{sheafificationThm}[subsection]{Sheafification Theorem}
  \newtheorem{subtheorem}{Theorem}[subsection]
  \newtheorem{subproposition}[subtheorem]{Proposition}
  \newtheorem{sublemma}[subtheorem]{Lemma}
  \newtheorem{subcorollary}[subtheorem]{Corollary}
  \newtheorem{subnothing}[subtheorem]{}

\theoremstyle{definition}
  \newtheorem{definition}[subsection]{Definition}
  \newtheorem{definitions}[subsection]{Definitions}
  \newtheorem*{definonumber}{Definition}
  \newtheorem{nothing*}[subsection]{}
  \newtheorem{example}[subsection]{Example}
  \newtheorem{examples}[subsection]{Examples}
  \newtheorem*{solution}{Solution}
  \newtheorem*{exnonumber}{Example}
  \newtheorem*{quesnonumber}{Question}
  \newtheorem{question}{Question}
  \newtheorem{problem}[subsection]{Problem} 
  \newtheorem{exercise}[subsection]{Exercise} 
  \newtheorem{notation}[subsection]{Notation}
  \newtheorem{notations}[subsection]{Notations}
  \newtheorem{step}{Step}
  \newtheorem*{claim}{Claim}
  \newtheorem{assumptions}[subsection]{Assumptions}
  \newtheorem{subdefinition}[subtheorem]{Definition}
  \newtheorem{subnotation}[subtheorem]{Notation}
  \newtheorem{subnotations}[subtheorem]{Notations}
  \newtheorem{subexample}[subtheorem]{Example}
  \newtheorem{subnothing*}[subtheorem]{}

\theoremstyle{remark}
  \newtheorem*{remark}{Remark}
  \newtheorem*{remarks}{Remarks}
  \newtheorem*{warning}{Warning}
  \newtheorem*{smallexample}{Example}


%%%%%%%%%%%%%%%%%%%%%%%%%%%%%%%%%%%%%%%%%%%%%%%%%%%%%%%%%
%% Macros (Most acronyms come from french)
%%%%%%%%%%%%%%%%%%%%%%%%%%%%%%%%%%%%%%%%%%%%%%%%%%%%%%%%%

\DeclareMathOperator*{\Oplus}{\oplus} 

\newcommand{\coker}{    \operatorname{{\rm coker}}}
\newcommand{\Aut}{    \operatorname{{\rm Aut}}}
\newcommand{\Spec}{   \operatorname{{\rm Spec}}}
\newcommand{\Proj}{   \operatorname{{\rm Proj}}}
\newcommand{\rank}{   \operatorname{{\rm rank}}}
\newcommand{\Reg}{    \operatorname{{\rm Reg}}}
\newcommand{\haut}{   \operatorname{{\rm ht}}}
\newcommand{\supp}{   \operatorname{{\rm supp}}}
\newcommand{\image}{  \operatorname{{\rm im}}}
\newcommand{\ord}{    \operatorname{{\rm ord}}}
\newcommand{\bideg}{  \operatorname{{\rm bideg}}}
\newcommand{\trdeg}{  \operatorname{{\rm trdeg}}}
\newcommand{\Frac}{   \operatorname{{\rm Frac}}}
\newcommand{\Char}{   \operatorname{{\rm char}}}
\newcommand{\Span}{   \operatorname{{\rm Span}}}
\newcommand{\Sing}{   \operatorname{{\rm Sing}}}
\newcommand{\Pic}{    \operatorname{{\rm Pic}}}
\newcommand{\Cl}{   \operatorname{{\rm Cl}}}
\newcommand{\dom}{    \operatorname{{\rm dom}}}
\newcommand{\codom}{  \operatorname{{\rm codom}}}
\renewcommand{\div}{  \operatorname{{\rm div}}}
\newcommand{\id}{   \operatorname{{\rm id}}}
\newcommand{\ob}{   \operatorname{{\rm ob}}}
\newcommand{\Hom}{    \operatorname{{\rm Hom}}}
\newcommand{\Set}{    \operatorname{{\rm\bf Set}}}
\newcommand{\Top}{    \operatorname{{\rm\bf Top}}}
\newcommand{\Grp}{    \operatorname{{\rm\bf Grp}}}
\newcommand{\CRng}{   \operatorname{{\rm\bf CRng}}}

\newcommand{\Mod}[1]{\mbox{\rm ${#1}$-\bf Mod}}
\newcommand{\setspec}[2]{\big\{\,#1\, \mid \,#2\, \big\}}
\newcommand{\powerset}{\raisebox{\depth}{\Large $\wp$}}

\newcommand{\notdiv}{\not\hspace{\mylength}\mid}
\newcommand{\epi}{\twoheadrightarrow}
\newcommand{\overepi}[1]{\overset{ #1 }{\epi}}
\newcommand{\monic}{\rightarrowtail}
\newcommand{\overmonic}[1]{\overset{ #1 }{\monic}}
\newcommand{\Integ}{\ensuremath{\mathbb{Z}}}
\newcommand{\Nat}{\ensuremath{\mathbb{N}}}
\newcommand{\Rat}{\ensuremath{\mathbb{Q}}}
\newcommand{\Comp}{\ensuremath{\mathbb{C}}}
\newcommand{\Reals}{\ensuremath{\mathbb{R}}}
\newcommand{\aff}{\ensuremath{\mathbb{A}}}
\newcommand{\proj}{\ensuremath{\mathbb{P}}}
\newcommand{\bk}{{\ensuremath{\rm \bf k}}}
\newcommand{\ck}{{\bar{\bk}}}
\newcommand{\kk}[1]{\bk^{[#1]}}

\newcommand{\Aeul}{\EuScript{A}}
\newcommand{\Beul}{\EuScript{B}}
\newcommand{\Ceul}{\EuScript{C}}
\newcommand{\Deul}{\EuScript{D}}
\newcommand{\Eeul}{\EuScript{E}}
\newcommand{\Feul}{\EuScript{F}}
\newcommand{\Geul}{\EuScript{G}}
\newcommand{\Heul}{\EuScript{H}}
\newcommand{\Keul}{\EuScript{K}}
\newcommand{\Oeul}{\EuScript{O}}
\newcommand{\Peul}{\EuScript{P}}
\newcommand{\Seul}{\EuScript{S}}

\newcommand{\Acal}{\mathcal{A}}
\newcommand{\Bcal}{\mathcal{B}}
\newcommand{\Ccal}{\mathcal{C}}
\newcommand{\Dcal}{\mathcal{D}}
\newcommand{\Ecal}{\mathcal{E}}
\newcommand{\Fcal}{\mathcal{F}}
\newcommand{\Gcal}{\mathcal{G}}
\newcommand{\OV}{\mathcal{O}}

\newcommand{\pgoth}{\mathfrak{p}}
\newcommand{\Pgoth}{\mathfrak{P}}
\newcommand{\qgoth}{\mathfrak{q}}
\newcommand{\Qgoth}{\mathfrak{Q}}
\newcommand{\m}{\mathfrak{m}}
\newcommand{\M}{\mathfrak{M}}

%%% DEAL With these later
\newcommand{\Q}{\mathbb{Q}}
\newcommand{\R}{\mathbb{R}}
\newcommand{\Z}{\mathbb{Z}}
\newcommand{\C}{\mathbb{C}}
\newcommand{\K}{\mathbb{K}}
\newcommand{\N}{\mathbb{N}}
\newcommand{\FP}{F_P}
\newcommand{\PF}{\mathbb{P}_F}
\newcommand{\LL}{\mathscr{L}} %% RRP 
\newcommand{\Leul}{\mathscr{L}} %% Dimension  of the RRP
\newcommand{\ldim}{\EuScript{l}}
\newcommand{\A}{\mathcal{A}_F}
\newcommand{\notzero}{\backslash \lbrace 0 \rbrace }
\newcommand{\IDEALa}{\mathfrak{a}}

\newcommand{\IDEALp}{\mathfrak{p}}
\newcommand{\fX}{f(X) = a_0 + a_1X + ... + a_nX^n}
\newcommand{\fx}{f(x) = a_0 + a_1x + ... + a_nx^n}
\newcommand{\ket}{\big \rangle}
\newcommand{\bra}{\big \langle}

\newcommand{\Abf}{\mathbf{A}}
\newcommand{\Bbf}{\mathbf{B}}
\newcommand{\Cbf}{\mathbf{C}}
\newcommand{\Dbf}{\mathbf{D}}
\newcommand{\Ebf}{\mathbf{E}}

\newcommand{\PP}{\mathbb{P}}
\newcommand{\PPP}{\mathbb{P}}
\newcommand{\SSS}{\mathbb{S}}
\newcommand{\VV}{\mathbb{V}}
\newcommand{\VVV}{\mathbb{V}}

\newcommand{\dirlim}{\varinjlim}
\newcommand{\ssi}{\Leftrightarrow}
\newcommand{\isom}{\cong}
\renewcommand{\epsilon}{\varepsilon}
\renewcommand{\phi}{\varphi}
\renewcommand{\emptyset}{\varnothing}
\newcommand{\rien}[1]{}
\renewcommand{\baselinestretch}{1.07}
\newcommand{\HRule}{\rule{\linewidth}{0.5mm}} 

%%%%%%%%%%%%%%%%%%%%%%%%%%%%%%%%%%%%%%%%%%%%%%%%%%%%%%%%%
% Pages sizing
%%%%%%%%%%%%%%%%%%%%%%%%%%%%%%%%%%%%%%%%%%%%%%%%%%%%%%%%%

% \setlength{\textwidth}{15.5cm}
% % \setlength{\mylength}{1.45\mylength}
% % \settowidth{\mylength}{$\,$}
% \addtolength{\oddsidemargin}{-1cm}
% \addtolength{\evensidemargin}{-1cm}
% \addtolength{\textheight}{14mm}

% \raggedbottom
% \CompileMatrices
% \newlength{\mylength}

\setlength\parindent{0pt}

%% Assignment info 
\begin{document}
{\huge Abelian Variety Cryptosystems} \\
{\large Kevin Johnson 6017605} \\
{\large \today} \\
\HRule  % Horizontal line

\section{The Discrete Logarithm Problem}
\section{Polynomials, Algebraic Sets and Genus}
\section{Elliptic Curves}
\section{Hyperelliptic Curves}
  Unlike elliptic curves, when the genus $\mf{g}$ of a curve $\mf{C} $ is greater than $1$, the set of points on $\mf{C}$ will not always form a group. 

\begin{example}

\end{example}

\subsection{The Jacobian of a Hyperelliptic Curve}

Luckily, there is another way to form an abelian group with hyperelliptic curves. Indeed, let $\mf{D}$ be the set of all formal finite sums 

$$ \sum_i m_i P_i $$ 

where $m_i \in \mathbb{Z}$ and $P_i$ are points on the curve $\mf{C}$. We call elements of $\mf{D}$ divisors of $\mf{C}$. Given a rational function $f$ in $\mathbb{Z}_p[\mf{C}]$, we can define the corresponding divisor to $f$ as

$$(f) = \sum_i m_i P_i $$ where $P_i$ are the zeros and poles of $f$ with multiplicities $m_i$. 

\begin{example}
\end{example} 

Divisors of this form are called principal divisors and we let $\mf{P}$ denote the subset of all of them in $\mf{D}$. If we define the operation on $\mf{D}$ by 

$$ \sum_i m_i P_i  + \sum_i m^\prime_i P_i  = \sum_i (m_i+m^\prime) P_i $$ 

then $\mf{D}$ becomes and abelian group. Unfortunetly, this group is far too large and unstructured for cryptographic purposes. So we consider the subgroup $\mf{D}^0$ of all divisors of $\mf{D}$ whos coefficients sum to $0$. That is, divisors $ \sum_i m_i P_i $ such that $\sum_i m_i = 0$. \\ 

This subgroup is still infinite, but that can be remedied by defining two divisors $D_1, D_2$ of $\mf{D}^0$ to be equal if $D_1 - D_2$ is equal to the divisor of a rational function on $\mf{C}$. That is, $D_1 - D_2 = (f) $ for $f \in \mathbb{Z}_p[\mf{C}]$. This new quotient group, denoted $$\mf{J} = \mf{D}^0 / \mf{P}$$ is called the jacobian of the curve $\mf{C}$ and is a finite cyclic group. This will be the group used to build hyperelliptic cryptosystems.

\subsection{Representation of Divisors}

Athough the Jacobian $\mf{J}$ of an hyperelliptic curve $\mf{C}$ is a finite abelian group, elements of $\mf{J}$ are very hard to represent. 

\begin{example}
\end{example} 

% Definition of semi-reduced divisor and how to find it and talk about the weight of a divisor y1report.pdf 

To make the group operation in $\mf{J}$ tractable, we ustilize the mumford representation of a divisor which is described as follows. Let $D$ be a semi-reduced with points $P_i = (x_i,y_i)$. We associate to $D$ polynomials $a,b \in \mathbb{Z}_p[x]$ such that $$a(x) = \prod^r_i (x - x_i) $$ $$ b(x_i) = y_i \text{ } 1 \leq i \leq r $$ where $\deg b < \deg a$ and $(x - x_i)^{k_i} \mid b - y_i$, if $k_i$ is the multiplicity of $P_i$. Denote this representation $D \stackrel{\text{def}}{=} \text{div} (a,b)$.

\subsection{The Group Law}

The group operation can be divided into two parts - \textit{Composition} and \textit{reduction} as described in [\ref{TanjaLange}]. \\ 	


\large{\textbf{Composition}}

Given two divisors represented as $D_1= \text{div}(a_1,b_1), D_2 = \text{div} (a_2,b_2) $ 

\begin{enumerate}[1.]
	\item compute $d_0 = \text{gcd}(a_1,a_2)$ and find the unique $c_1,e_1 \in \mathbb{Z}_p[x]$ such that $d_0 = c_1a_1 + e_1a_2$ 
	\item compute $d = \text{gcd}(d_1,b_1 + b_2)$ and find the unique $c_2, e_2 \in \mathbb{Z}_p[x]$ such that $d = c_2d_1 + e_2(b_1 + b_2)$ 
	\item compute $a_3 = \frac{a_1,a_1}{d^2}$
	\item compute $b_3 = \frac{c_2c_2a_1 + c_2e_1a_2 + e_2(b_1b_2 + f)}{d} \text{ mod } \frac{a_1,a_1}{d^2}$
	\item \label{repeat} compute $a_3^\prime = \frac{f - b_3^2}{a_3}$ and $b_3^\prime = - b_3$ mod $a_3^\prime$ 
	\item while $\deg (a_3^\prime ) > g$, reassign $a_3 = a_3^\prime, b_3 = b_3^\prime$ and repeat step \ref{repeat}
	\item divide $a_3^\prime $ by its leading coefficient so that $a_3^\prime $ becomes monic
	\item the output $div(a_3^\prime,b_3^\prime) = D_1 + D_2$ 
\end{enumerate}


Why Does this work? 

\begin{example}
\end{example}





\section{Abelian Varieties}


\end{document}




