%TODO
%Proper definition of algebraic variety
%Definition of Dimension 
%Definition of Genus
%Start with algebraically closed field

When working with polynomials in more than one variable, it is sometimes simpler to use the formalism of classic algebraic geometry as found in [\ref{Hartshorne}]. Let $K$ be an arbitray field (for the purposes of this report we will only consider $K = \mathbb{Z}_p$, where $p>3$).

% there should be some mention of k rational points

Let $A = K[X_1,...,X_n]$ represent the polynomial ring in $n$ variables over $K$. Given a subset $T \subseteq A $, we may define $$Z(T) \stackrel{\text{def}}{=} \lbrace x \in K^n \mid f(x) = 0 \text{ for all } f \in T \rbrace $$ to be the set of all common zeros of polynomials in $T$. We call $Y \subseteq K^n $ an \textit{algebraic subset} if $Y = Z(T)$ for some subset $T \subseteq A$. Similary, given an subset $Y \subseteq K^n$, may define the set of A $$ I(Y) \stackrel{\text{def}}{=} \lbrace f \in A \mid f(x) = 0 \text { for all } x \in Y \rbrace $$ In fact, $I(Y)$ is an ideal in $A$ often called \textit{the ideal of Y}. So we may consider the quotient ring $$A(Y) \stackrel{\text{def}}{=} A / I(Y) $$ We refer to this ring as the \textit{the coordinate ring of Y}.  

% \subsection{Dimension}
% \subsection{Genus}


