%TODO
%Proper definition of algebraic variety
%Definition of Dimension 
%Definition of Genus
%Start with algebraically closed field


In this report we focus on groups which arise from the solution set of polynomial equations over finite fields. The most famous of these groups is the set of points on an elliptic curve characterized by the equations $y^2 = x^3 + ax + b$ where $a,b \in \mathbb{F}_p$ satisfy $4a^3 + 27b^2 \not\equiv 0 \text{ mod } p$. In recent years, groups arising from these equations have found significant success in public key cryptography even though it is an open questions whether these kind of groups are actually cryptographically secure. One of the benefits of using elliptic curves groups is that the fastest known attack on them has $O(\sqrt{N})$ complexity, where $N$ is the order of the group. This means significantly smaller keys can be used in comparison to the key length of RSA. A natural question is 

$$
\textbf{What about other polynomial equations?}
$$ 

It turns out that there are infinitely many groups which arise from the solution sets of polynomial equations. Which of these groups are cryptographically viable is an active area of research. First we develope some terminology required to describes these groups. \\ 

Let $K$ be a field. Let $K[x_1,...,x_n]$ represent the polynomial ring in $n$ variables over $K$. Given a subset $T \subseteq K[x_1,...,x_n] $, we may define 

$$
Z(T) = \lbrace P \in K^n \mid f(P) = 0 \text{ for all } f \in T \rbrace 
$$ 

to be the set of all common zeros of polynomials in $T$. We call $Y \subseteq K^n $ an \textit{algebraic subset} if $Y = Z(T)$ for some subset $T \subseteq K[x_1,...,x_n]$. We say a algebraic set $Y$ is reducible if it can be written as the union of two smaller algebraic sets. For example, let $Y = Z(x^2 - yz, xz-x)$ as a subset of $\mathbb{C}^3$. That is, $Y$ is the set of complex valued points in $\mathbb{C}^3$ which satisfy each of the equations $x^2 - yz$ and $xz-x$. Notice that

\begin{align*}
	Y &= Z(x^2 - yz, xz-x) \\
	&= Z(x^2 - yz, x(z-1)) \\
	&= Z(x^2 - yz,x) \cup Z(x^2 - yz,z-1) \\
	&= Z(yz,x) \cup Z(x^2 - y,z -1) \\
	&= Z(y,x) \cup Z(z,x) \cup Z(x^2 - y, z - 1)
\end{align*} 

So $Y$ is reducible in $\mathbb{C}^3$. If an algebraic set is not reducible, it is called irreducible. 


% Similary, given an subset $Y \subseteq K^n$, may define the set of A 

% $$ 
% I(Y) = \lbrace f \in A \mid f(P) = 0 \text { for all } P \in Y \rbrace 
% $$ 

% In fact, $I(Y)$ is an ideal in $A$ often called \textit{the ideal of Y}. So we may consider the quotient ring 

% $$
% A(Y) = A / I(Y) 
% $$ 

% We refer to this ring as the \textit{the coordinate ring of Y}. This ring encodes a lot of information about an alebraic set. 





\subsection{Dimension}
\subsection{Genus}


